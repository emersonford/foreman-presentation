\documentclass{standalone}

\usepackage[utf8]{inputenc}
\usepackage{pgf-umlsd}

% new an instance thread
% Example:
% \newinst[color]{var}{name}{class}
\newcommand{\newspacedthread}[3][none]{
  \newinst[2]{#2}{#3}
  \stepcounter{threadnum}
  \node[below of=inst\theinstnum,node distance=0.8cm] (thread\thethreadnum) {};
  \tikzstyle{threadcolor\thethreadnum}=[fill=#1]
  \tikzstyle{instcolor#2}=[fill=#1]
}

\begin{document}

\begin{sequencediagram}
	\tikzstyle{threadstyle}+=[draw=none,fill=none]
	\newthread[none]{A}{Remote User}{}
	\newspacedthread{B}{Remote Host}{}
	\newinst[1]{C}{Foreman Master}{}
	\newinst[1]{D}{Smart Proxy}{}
	\newinst[1]{E}{Remote DHCP}{}

	\postlevel
	\begin{messcall}{A}{\shortstack{Insert Boot USB and \\
				boot host into USB}}{B}{}
	\end{messcall}


	\draw[line width=.1mm,double distance=3pt] ([xshift=70pt] 0,-3.5 -| current page.west) -- node[fill=white] {Boot into iPXE USB Image} ([xshift=-40pt] 0,-3.5 -| current page.east);

\end{sequencediagram}

\end{document}
