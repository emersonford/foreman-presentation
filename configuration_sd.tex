\documentclass{standalone}

\usepackage[utf8]{inputenc}
\usepackage{pgf-umlsd}

% new an instance thread
% Example:
% \newinst[color]{var}{name}{class}
\newcommand{\newspacedthread}[3][gray!30]{
  \newinst[6]{#2}{#3}
  \stepcounter{threadnum}
  \node[below of=inst\theinstnum,node distance=0.8cm] (thread\thethreadnum) {};
  \tikzstyle{threadcolor\thethreadnum}=[fill=#1]
  \tikzstyle{instcolor#2}=[fill=#1]
}

\begin{document}

\begin{sequencediagram}
	% \tikzstyle{threadstyle}+=[draw=none,fill=none]
	\newthread{B}{Remote Host}{}
	\newinst[6]{E}{Smart Proxy}{}
	\newinst[6]{D}{Foreman Master}{}

	\draw[line width=.1mm,double distance=3pt] ([xshift=70pt] 0,-1.5 -| current page.west) -- node[fill=white] {Remote Host in Installed OS} ([xshift=20pt] 0,-1.5 -| current page.east);

	\postlevel
	\postlevel

	\begin{call}{B}{Register Host Build Done}{E}{}
		\begin{call}{E}{Register Host Build Done in DB/web/cli}{D}{}
		\end{call}
	\end{call}

	\postlevel

	\begin{sdblock}{Loop}{Puppet Agent Loop}
		\begin{call}{B}{Run `puppet agent`}{B}{}
			\postlevel
			\begin{call}{B}{Fetch Puppet manifests}{E}{Manifest file}
				\begin{call}{E}{Fetch manifests from DB}{D}{Manifest file}
				\end{call}
			\end{call}

			\begin{call}{B}{Apply manifest}{B}{}
			\end{call}
			\postlevel
		\end{call}
	\end{sdblock}

\end{sequencediagram}

\end{document}
